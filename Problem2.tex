\documentclass[12pt]{article}
\usepackage[utf8]{inputenc}
\usepackage{graphicx}
\usepackage{amsmath}


\begin{document}

\begin{center}\textbf{Problem 2- Interview}\end{center}
Interviewee- Dr. Parveen Lata\newline
She is an Associate Professor in Department of Basic and Applied Sciences in Punjabi University. She has done M.Phil in Number Theory and Applied Mathematics. She has done her PhD in problems in Micropolar Thermoelastic media. She has almost 20 years of teaching experience in the field of Mathematics.\newline\newline
Q1- What field are you researching in Mathematics?\newline
A1- Applied Mathematics, Topology and Algebra.\newline\newline
Q2- Are there other fields in mathematics you are interested in?\newline
A2- I have interests in Geometry and Number systems.\newline\newline
Q3- How long have you been working in this field?\newline
A3- More than 20 Years.\newline\newline
Q4- Have you heard of some special irrational constants in the field of Mathematics? Can you give some examples?\newline
A4- I have researched on some irrational constants like π,e,Khinchin's Constant (K).\newline\newline
Q5-Have you heard of the Irrational constant "Golden Ratio($\phi$)"?\newline
A5- Yes. I have studied about the golden ratio during my research.\newline\newline
Q6- What is the Golden Ratio?\newline
A6- Two quantities are in the golden ratio if their ratio is the same as the ratio of their sum to the larger of the two quantities. For quantities a and b such that $>$ b $>$ 0,\newline\newline
\begin{center}{\displaystyle {\frac {a+b}{a}}={\frac {a}{b}}\ {\stackrel {\text{def}}{=}}\ $\phi$}\end{center}\newline
Q7-How do you derive its value?\newline
A7- It is the solution to the quadratic equation $x^2-x-1=0$.\newline\newline\newline
Q8- What is the value of Golden Ratio?\newline
A8- The approximate value of golden ratio is 1.618\newline\newline
Q9-What do you think are the applications of the Golden Ratio in mathematics?\newline
A9- The golden ratio is used mostly in the Geometry to create designs that are in proportions and are pleasing to the eye. It is not used as such in Mathematics directly but even the ratio of consecutive numbers in fibonacci series are close to the golden ratio.\newline\newline
Q10- Are there any other fields outside of Mathematics that use the Golden Ratio?\newline
A10- Golden ratio is used extensively in Architecture, Art, Plastic Surgery to name a few. Golden ratio appears in surprising frequency in the nature as well.\newline\newline
Q11- Could you give me some examples of its use in architecture or art?\newline
A11- The Great Pyramids of Gaza, Parthenon in Athens, Michelangelo’s The Creation of Adam on the ceiling of the Sistine Chapel and Da Vinci’s Mona Lisa are some of the famous examples that use the Golden Ratio. Even the famous painting "The Last Supper" uses Golden ratio many times.\newline\newline
Q12- Why is the golden ratio so pleasing to the eye?\newline
A12- Our brain unknowingly prefers objects that follow the golden ratio. Even the plastic surgeons follow the golden ratio during facial surgeries so that the results look natural. It’s a subconscious attraction and even tiny tweaks that make an anything closer to the Golden Ratio have a large impact on our brains.\newline\newline
Q13- What device do you generally use for complex calculations in your field?\newline
A13- I generally use a scientific calculator.\newline\newline
Q14- Would you like to include Irrational constants like Golden ratio in the calculator?\newline
A14- Yes. I would prefer if the calculators could directly provide the value for such irrational constants which are used frequently in my research.\newline\newline
Q15- What kind of interface would you like for the device? Would you prefer a desktop Application, Web Application or a mobile Application?\newline
A15- I would prefer a Mobile or desktop application with an easy to use interface.\newline


\end{document}